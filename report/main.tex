\documentclass[12pt,a4paper]{article}
\usepackage[T1]{fontenc}
\usepackage[utf8]{inputenc}
\usepackage{geometry}
\geometry{tmargin=1in,bmargin=1in,lmargin=1.2in,rmargin=1.2in}
\usepackage[english]{babel}
\usepackage{amsmath}
\usepackage{amssymb}
\usepackage{amsthm}
\usepackage{booktabs}
\usepackage{graphicx}
\usepackage{float}
\usepackage{ae,aecompl}
\usepackage[multiple]{footmisc}
\usepackage[hyperfootnotes=false]{hyperref}
\usepackage{bookmark}
\hypersetup{colorlinks=true,citecolor=blue}
\usepackage{MJOARTI}
\usepackage{setspace}
\usepackage{rotating}
\usepackage{lscape}
\usepackage{pdflscape}
\usepackage{subfigure}
\usepackage{caption}
\usepackage{ifthen}
\usepackage{url}
\usepackage{blkarray}
\usepackage{multirow}
\usepackage{tabularx}
\usepackage{longtable}
\usepackage{enumerate}
\usepackage{supertabular}
\usepackage{fancyhdr}
\usepackage{epstopdf}
\usepackage{multicol}
\usepackage{titlesec}
\usepackage{eurosym}
\usepackage{arydshln}
\usepackage{natbib}
\usepackage{threeparttable}
\usepackage{comment}
\newcommand\citeapos[1]{\citeauthor{#1}'s (\citeyear{#1})}
\newtheorem{theorem}{Theorem}
\newtheorem{algorithm}{Algorithm}
\newtheorem{axiom}{Axiom}
\newtheorem{case}{Case}
\newtheorem{claim}{Claim}
\newtheorem{conclusion}{Conclusion}
\newtheorem{assumption}{Assumption}
\newtheorem{hypothesis}{Hypothesis}
\newtheorem{condition}{Condition}
\newtheorem{conjecture}{Conjecture}
\newtheorem{corollary}{Corollary}
\newtheorem{criterion}{Criterion}
\newtheorem{definition}{Definition}
\newtheorem{example}{Example}
\newtheorem{exercise}{Exercise}
\newtheorem{lemma}{Lemma}
\newtheorem{notation}{Notation}
\newtheorem{problem}{Problem}
\newtheorem{proposition}{Proposition}
\newtheorem{remark}{Remark}
\newtheorem{solution}{Solution}
\newtheorem{summary}{Summary}
\newtheorem{class}{Class}
\usepackage{array}
\newcolumntype{k}{>{\centering\arraybackslash}p{2cm}}
\usepackage{ltcaption}
\usepackage{color}
\usepackage{textcomp}
\usepackage{ragged2e}
\floatplacement{table}{t}

\definecolor{dark-red}{rgb}{0.6,0,0}
\definecolor{listinggray}{gray}{0.9}
\definecolor{darkgreen}{rgb}{0,0.4,0}
\definecolor{lbcolor}{rgb}{0.9,0.9,0.9}


%\setlength{\absleftindent}{-0.25in} % For Restart
%\setlength{\absrightindent}{-0.25in} % For Restart

%\renewcommand\thesection{\Roman{section}.}
%\renewcommand\thesubsection{\Alph{subsection}.}

\begin{document}

\date{February 2026}

\title{
\textbf{Replication Report} \\[0.25cm]
\textbf{A comment on ``Firm Donations and Political Rhetoric: Evidence from a National Ban''}%
%
 \thanks{
Authors:
Andriyanto: University of Queensland. E-mail: \href{mailto:andriyanto@uq.edu.au}{andriyanto@uq.edu.au} (corresponding author).
Eugster, Nicolas: University of Queensland. E-mail: \href{mailto:nicolas.eugster@gmail.com}{nicolas.eugster@gmail.com}.
Huang, Guan-Jia: University of Queensland. E-mail: \href{mailto:guanjia.huang@student.uq.edu.au}{guanjia.huang@student.uq.edu.au}.
Kalaycı, Kenan: University of Queensland. E-mail: \href{mailto:k.kalayci@uq.edu.au}{k.kalayci@uq.edu.au}.
We declare no conflict of interest with the original authors. See I4R's conflict of interest policy here: \url{https://i4replication.org/conflict.html}.}
}
\author{Andriyanto, Nicolas Eugster, Guan-Jia Huang, Kenan Kalaycı}

\maketitle

\renewcommand{\abstractname}{Abstract}
\begin{abstract}

%---------------------------------------------------------------------------------%
%INSTRUCTIONS:
%Summarize in few sentences the original study, focusing on the main results in the original abstract in terms of word claim  which you attempt to reproduce or replicate. Provide information, if relevant, on the magnitude and statistical significance of the main results. Then report all your reproduction and replication results. 
 

We replicate the key findings of \cite{Cage}, who investigate whether France's 1995 ban on corporate donations influenced politicians' campaign communication. The study uses a difference-in-differences approach on a dataset combining donations received by parliamentary candidates with their individual campaign manifestos. We obtain the replication package from the AEJ: Economic Policy ICPSR repository and use Stata to reproduce the main results, excluding the Python-based text analysis pipeline.

In terms of computational reproducibility, we successfully reproduce all main tables (Tables~3--8, Panel~A) from the analysis data using the authors' Stata code. No coding errors were found. In terms of robustness, we conduct three sets of checks. First, winsorizing firm donations at the 1st and 99th percentiles yields estimates closely aligned with the baseline, confirming that outliers do not drive the results. Second, we vary the set of control variables (no controls, no candidate controls, decomposed mandates, total mandate count), and the core findings on local campaigning are stable across all specifications. Third, restricting the sample to candidates who received positive firm donations in 1993 confirms the negative effect on local emphasis ($-0.118$, $p<0.01$), though the positive effect on national references becomes statistically insignificant. Overall, we assess the main findings as robust and well-supported.
\newline

\textsc{Keywords}: campaign finance, corporate donations, political rhetoric, difference-in-differences, France \newline

\textsc{JEL codes}: D72, P16
\end{abstract}

\clearpage


%%---------------------------------------------------------------------------------
%\section*{Introduction: Instructions}
%\begin{footnotesize}
%\begin{spacing}{1.0}
%Briefly describe the main data sources, method, policy or treatment, time period and population for which the estimates apply. Then describe the main scientific claims (descriptive or causal) and robustness checks if those that are relevant for your re-analysis or replication. Quote the original part of the study that has the main scientific claim(s) including page number(s). As suggested in the Guide for Accelerating Computational Reproducibility in the Social Sciences (https://bitss.github.io/ACRE/), structure your summary of the main findings and methodology as follows: "The paper tested the effect of X on Y for population P, using method M. The main results show an effect of magnitude E (specify units and standard errors)" or "The paper estimated the value of Y (estimated or predicted) for population P under dimensions X using method M. The main results presented an estimate of magnitude E (specify units and standard errors)". This template assumes that the paper’s scientific claims are focused on estimating a causal relationship. In the event that the original study is estimating/predicting a descriptive statistic of a population, or something else, then describe and quote the results accordingly using precise claims from the original study. \\

%How to select claims to reproduce/replicate? There are three possibilities; (1) select claims for all "hypotheses tests" in the original study, (2) select claims mentioned in the abstract or (3) select claims for what is considered the main result in the paper as stated by the original author(s). For the last option, provide a quote from the original study confirming that the claim chosen is considered the main result by the original author(s). \\      

%Next, summarize your reproduction and/or replication. Start by stating how you have obtained the data and codes and if the original author(s) answered your request(s) and questions. Indicate the repository where your programs and data are located. Then proceed with a description of your computational reproducibility by describing if you have found coding error(s) and how they affect the main conclusions. \\

%For robustness reproduction, clearly state your robustness checks and how they affect the main point estimates. For replication, clearly describe the new data. \\

%For all reproduction/replication types, be precise and summarize your results as follows: ``Implementing this robustness increases/decreases the size of the main point estimate for outcome Y by X and the estimate is not anymore statistically significant at the X\% level'' or ``Implementing this robustness check has no effect on the magnitude or the statistical significance of the main point estimate.'' Also report the coefficient (or other effect size), the standard error of the coefficient/effect size, the test statistic including degrees of freedom if relevant, and the p-value for all tests. \\

%Uphold a professional and constructive tone throughout the reproduction report, fostering a collaborative and supportive environment for the reproduction process.
%In case you have questions while going through this template, please do not hesitate to contact the I4R team. \\
%\end{spacing}
%\end{footnotesize}

%\clearpage 
%%---------------------------------------------------------------------------------


\section{Introduction}
\begin{spacing}{1.5}
\cite{Cage}, henceforth CLM, investigate whether the French ban on corporate donations to political campaigns, enacted in 1995, influenced politicians' campaign communication. The setting is French parliamentary elections: the authors combine digitized campaign finance records with text-analyzed campaign manifestos from the 1993 (pre-ban) and 1997 (post-ban) elections, exploiting cross-sectional variation in candidates' pre-ban reliance on corporate funding.

CLM employ a difference-in-differences approach comparing candidates who lost more versus fewer corporate donations after the ban. The main data set covers 2,602 candidate-election observations from the 1993 and 1997 legislative elections. CLM describe their main results as follows: banning corporate donations causes politicians who previously benefited to ``de-emphasize their local presence in their campaign communication, and to favor national politics instead'' (p.~218). Specifically, a one-standard-deviation loss in corporate donations decreases the local index by 15.8\% of a standard deviation ($p<0.01$). The ban also decreases the prevalence of local economic issues and pushes candidates to address other policy topics. On average, it does not shift discourse ideologically, but it pushes niche-party candidates toward more extreme rhetoric. Importantly, no comparable effects are found on legislative activity (written questions, debate interventions), suggesting that the changes are campaign-specific.

In the present report prepared for the Institute for Replication, we investigate whether CLM's analytical results are computationally reproducible and test their robustness to: (1)~winsorizing the treatment variable to limit the influence of outliers, (2)~varying the set of control variables, and (3)~restricting the sample to candidates who received positive firm donations before the ban. We obtained the replication package from the AEJ: Economic Policy page on the ICPSR repository, which contains raw and analysis data as well as the full cleaning and analysis code in Stata. We did not replicate the Python-based text analysis pipeline that generates the outcome variables from digitized manifestos. Our replication code is available at \url{https://github.com/kenankalayci/ReplicationGames_Cageetal_2024}.

In terms of reproducibility, we successfully reproduce all main tables (Tables~3--8, Panel~A) using the authors' Stata code. No coding errors were found, and all results match the published paper.

In terms of robustness, our three sets of checks broadly support the original findings. Winsorizing firm donations at the 1st and 99th percentiles yields estimates closely aligned with the baseline. Varying the control set confirms that the core local-versus-national result is stable. Restricting to positive-donation candidates confirms the negative effect on local emphasis, though the positive effect on national references loses significance.
\end{spacing}

%\clearpage
%
%%---------------------------------------------------------------------------------
%\section*{Computational Reproducibility: Instructions}
%\begin{footnotesize}
%\begin{spacing}{1.0}
%Describe first the completeness of the replication package. Note whether there were ``No code'', ``Cleaning code (partial)'', ``Cleaning code (complete)'', ``Analysis code (partial)'' or ``Analysis code''. For the data, note whether there were ``No Data'', ``Raw Data (partial)'', ``Raw Data (complete)'', ``Analysis Data (partial)'' or ``Analysis Data (complete)''. \\

%Then describe which of the following best describes the computational reproducibility (i.e. duplicating the results using the provided code) of the original authors' paper:
%\begin{enumerate}
%\item \textit{Not Computationally Reproducible from Analysis Data}: analysis code with analysis data fails to run, or, produces results inconsistent with the paper.  
%\item \textit{Computationally Reproducible from Analysis Data}: analysis code with analysis data produces the same output as presented in the paper.  \
%\item \textit{Computationally Reproducible from Raw Data}: cleaning code and analysis code start from raw data and reproduce the output presented in the paper. \\
%\end{enumerate}

%Then, describe in detail if you have uncovered coding error(s) and how they affect the main conclusions. In the event that the sign, magnitude or statistical significance is changed for the main estimates or robustness checks, report the new point estimates (e.g., in a table) in this section. \\
	
%Fix the coding error(s) prior to conducting the rest of the analysis, but make sure to clearly state and disentangle the effect of the coding error(s) vs the change(s) made to the data and codes/procedures in the following sections. \\ 
%\end{spacing}
%\end{footnotesize}

%\clearpage
%%---------------------------------------------------------------------------------

\section{Computational Reproducibility}

\begin{spacing}{1.5}
We used the replication package here: \href{https://www.openicpsr.org/openicpsr/project/184946/version/V1/view}{hyperlink}. 
The replication package contains the raw and analysis data as well as the full cleaning and analysis code. 
Using these materials, we successfully reproduce all main results (Tables 3–8, Panel A) from the analysis data. 
Details are provided in Table \ref{tab:reppkg_table}.

\begin{table}[h]
\centering
\caption{Replication Package Contents and Reproducibility}
\label{tab:reppkg_table}
\footnotesize 
	\begin{tabular}{lccc}
	\hline \hline
	\\
	\textbf{Replication Package Item} 	& \textbf{Fully} & \textbf{Partial} & \textbf{No} \\
	\hline
	\\
	Raw data provided                 	& \checkmark &  &  \\ 
	Analysis data provided				& \checkmark &  &  \\ 
	\\
	Cleaning code provided				&\checkmark  &  &  \\ 
	Analysis code provided				& \checkmark &  &  \\ 
	\\
	Reproducible from raw data			& \checkmark  &  &  \\ 
	Reproducible from analysis data		& \checkmark &  &  \\ 
	\\
	\hline \hline
	\end{tabular}
\caption*{
\footnotesize \emph{Notes}: This table summarizes the replication package contents contained in \cite{Cage}. %\textbf{Displayed here only for replicators knowledge. Please put a similar table in your Tables appendix.}
}
\end{table}

\end{spacing}

%%---------------------------------------------------------------------------------
%\subsection*{Discrepancies Between Pre-analysis Plan and Article (Optional)}
%\begin{footnotesize}
%\begin{spacing}{1.0}
%
%Check whether the authors registered the study and published a pre-analysis plan (PAP). Please note, which of the following cases applies:
%\begin{itemize}
%\item No preregistration
%\item Preregistration without a PAP
%\item Preregistration with a PAP
%\end{itemize}
%
%If you have access to the PAP, you may compare it to the original paper and highlight deviations. Particularly, check for the following:
%\begin{itemize}
%\item Does the PAP specify the exact design of the study, the exact data collection and exactly how all analyses and tests will be conducted?
%\item Does the paper follow the pre-registered design and data collection?
%\item Does the paper carry out all analyses and tests reported in the paper exactly as specified in the PAP?
%\item Does the paper report all analyses and tests as specified in the PAP?
%\item Does the paper report additional analyses and tests not specified in the PAP?
%\end{itemize}
%
%For all deviations from the PAP, state whether the paper clearly mentions and explains this deviation. For example, does the paper mark deviations from the PAP as ``exploratory''?
%\end{spacing}
%\end{footnotesize}
%
\subsection{Pre-registration}
\begin{spacing}{1.5}
The original study was not pre-registered and no pre-analysis plan was filed.
\end{spacing}

%\clearpage
%%---------------------------------------------------------------------------------


%%---------------------------------------------------------------------------------
%\section{Robustness Reproduction and Replication with New Data: Instructions}
%\begin{footnotesize}
%\begin{spacing}{1.0}
%
%Clearly state/describe which type of reproduction/replication you are conducting. See definitions at the beginning of this document. For robustness reproduction, present your robustness checks and how they impact the main point estimates one by one so that it is clear how each modification to the specification/analysis impacts the main conclusions. Then you may combine them. Also, clearly state why you conduct each specific robustness check and/or modify the setting/model. \\
%
%Do not confuse general critiques of the original research with replication or robustness checks (Brown and Wood, 2018). For instance, any critique of the design or methodology (e.g., qualitatively discussing the validity of an exclusion restriction for an instrumental variable) should not be included unless it is to justify the validity of a robustness check (or using new data).
%
%Report all executed robustness checks, without any selective reporting (cherry-picking), to ensure a comprehensive and hence unbiased representation of the findings.
%
%\end{spacing}
%\end{footnotesize}
%
%%---------------------------------------------------------------------------------

\section{Robustness Reproduction and Replication}

\begin{spacing}{1.5}
We now turn to robustness reproduction, testing whether the main findings of \cite{Cage} are sensitive to plausible alternative analytical choices. We conduct three sets of robustness checks: (i)~winsorizing the treatment variable to limit the influence of outliers, (ii)~varying the set of control variables included in the regressions, and (iii)~restricting the sample to candidates who received positive firm donations before the ban. Throughout, we maintain the authors' core difference-in-differences specification with candidate and year$\times$party fixed effects, clustered standard errors at the constituency level, and the predicted 1993 donations as an instrument for treatment intensity.

\subsection{Winsorizing}
We follow the authors' difference-in-differences specification,
using firm donations winsorized at the 1st and 99th percentiles to limit the influence of extreme values. The results remain stable:
coefficients retain their sign and change only moderately in magnitude.
For instance, in Table \ref{tab:don_local_nat_winsor}, Panel B, the coefficient on National references increases from 0.098 to 0.145,
while the corresponding $p$-value moves from 0.078 to 0.037. The coefficient on the Local index moves from $-0.130$ to $-0.172$, remaining significant at the 1\% level. Similarly, the impact on Local references strengthens slightly from $-0.219$ to $-0.276$.

For partisan leaning (Table~\ref{tab:don_lr_orig_nat}), winsorization leaves the results essentially unchanged: the left-right score and extremeness coefficients remain small and statistically insignificant, consistent with the original finding that the ban does not systematically shift discourse ideologically.

The heterogeneity-by-party results (Table~\ref{tab:interaction_original}, Panel~B) are also robust. The Green party interaction coefficients---which are notably large---remain virtually identical under winsorization, as expected given that Green candidates received very few firm donations and the winsorization threshold does not bind for them. For the major parties (Socialist, Right), coefficients move slightly in magnitude but retain their sign and significance.

For broad policy topics (Table~\ref{tab:policyareas}), winsorization strengthens the economic policy coefficient from $-1.213$ to $-1.729$ while preserving its significance, but the social policy effect weakens slightly (from $p=0.024$ to $p=0.062$) and the homeland-and-administration effect loses significance ($p=0.275$). The foreign policy result remains marginally significant.

Finally, the legislative activity results (Tables~\ref{tab:questions} and~\ref{tab:interventions}) show that winsorization does not alter the key finding: firm donations affected campaign rhetoric but not legislative behaviour. The number of written questions and debate interventions remain statistically insignificant, as in the original.

Overall, the winsorized estimates closely align with the baseline, indicating that the main findings are not driven by outliers in the donation distribution.


\subsection{Alternative control specifications}

To assess sensitivity to the choice of covariates, we re-estimate the main specification (Table~3 of the original paper) under four alternative control configurations: (B)~no controls beyond the fixed effects, (C)~no candidate-specific controls, (D)~decomposing the ``other mandate'' variable into its three constituent roles (Conseiller d\'{e}partemental, S\'{e}nateur, and D\'{e}put\'{e} europ\'{e}en), and (E)~replacing the four individual mandate dummies with a single count of total mandates held. The results are reported in Table~\ref{tab:don_local_nat} (bottom panels).

Across all four specifications, the Local index and Local references coefficients remain negative and statistically significant at the 1\% level, with magnitudes ranging from $-0.113$ to $-0.142$ for the Local index and $-0.192$ to $-0.231$ for Local references. The National references coefficient is positive and significant under Panels B, D, and E, but loses marginal significance ($p>0.10$) when candidate-level controls are dropped entirely (Panel~C). This suggests that candidate characteristics partially absorb variation that otherwise manifests in the national-references outcome, but the core local-versus-national pattern remains robust.

The near-identical results between Panels A and D (decomposing mandates) and between A and E (count of mandates) confirm that the particular operationalization of the mandate control variable does not drive the findings.


\subsection{Restricting to positive-donation candidates}

A potential concern with the baseline specification is that it includes candidates who received zero firm donations in 1993, for whom the ``loss'' of corporate funding after the ban is mechanically zero. To address this, we restrict the sample to candidates who received strictly positive firm donations in 1993 ($N=1{,}246$, compared to $2{,}602$ in the full sample). This provides a cleaner test of whether the ban affected those who actually relied on corporate funding.

The results, reported in Table~\ref{tab:don_local_nat_positive}, confirm the main findings. The Local index coefficient is $-0.118$ (s.e.\ $0.032$, $p<0.01$), and Local references is $-0.230$ (s.e.\ $0.062$, $p<0.01$). However, the National references coefficient drops to $0.049$ and is no longer statistically significant (s.e.\ $0.056$, $p>0.10$). This suggests that the positive effect on national references is partially driven by the comparison with zero-donation candidates, while the decrease in local emphasis is robust to restricting the sample to actual corporate-donation recipients.

For partisan leaning outcomes (Table~\ref{tab:don_lr_positive}), the restricted sample yields similar null results: the left-right score coefficient is $-0.003$ ($p>0.10$) and extremeness is $0.006$ ($p>0.10$). The originality index shows a marginally significant negative effect ($-0.024$, $p<0.10$), which was not present in the full sample, though this is only borderline significant and should be interpreted cautiously given the reduced sample size. National party references is $0.009$ ($p>0.10$), consistent with the full-sample null result.

\medskip
Figure~\ref{fig:robustness} summarises the robustness of the Table~3 results visually, plotting the coefficient on firm donations (loss) across all seven specifications for each of the three main outcomes. The Local index and Local references estimates are consistently negative and significant across all specifications, while the National references coefficient is positive but loses significance under some specifications (no candidate controls, positive-donation restriction).

\end{spacing}
\section{Conclusion}
\begin{spacing}{1.5}
This report presents a comprehensive replication of \cite{Cage}, who study the effect of France's 1995 ban on corporate political donations on candidates' campaign rhetoric. We evaluate the computational reproducibility of their results and conduct a series of robustness checks.

Our main findings are as follows. First, the paper is fully computationally reproducible: all main results (Tables~3--8, Panel~A) can be replicated from the provided code and data without modification. The replication package is well-documented and complete, containing raw data, analysis data, cleaning code, and analysis code.

Second, the core finding---that losing corporate donations causes candidates to de-emphasize local issues in their campaign manifestos---is robust across all specifications we tested. Winsorizing the treatment variable at the 1st and 99th percentiles, varying the set of controls (including dropping all controls, dropping candidate controls, and alternative operationalizations of mandate variables), and restricting the sample to candidates who received positive firm donations in 1993 all yield qualitatively similar results for the Local index and Local references outcomes.

Third, some secondary results show greater sensitivity. The positive effect on National references, while significant in the baseline, loses significance when the sample is restricted to positive-donation recipients, suggesting that this particular result partly depends on the inclusion of zero-donation candidates in the comparison group. Similarly, the effects on broad policy topics (particularly social policy and homeland-and-administration) show some sensitivity to winsorization, though the economic policy result actually strengthens.

Fourth, the null finding on legislative activity---that corporate donations affect campaign rhetoric but not actual legislative behaviour---is fully robust across all our checks.

A limitation of both the original study and our replication is the inability to formally test the parallel trends assumption underlying the difference-in-differences design. Reliable donation data are only available for 1993 and 1997, and the 1988 pre-ban sample is too small for meaningful pre-trend analysis. Future replicators with access to additional archival data on pre-ban campaign finance could strengthen the causal interpretation of these results. Additionally, an independent replication of the text analysis pipeline (which uses Python-based NLP techniques to construct the outcome variables from digitized manifestos) would provide further confidence in the findings, as we relied on the pre-computed outcome variables from the replication package.

Overall, we assess the main claims of \cite{Cage} as well-supported by the data. The paper provides credible evidence that corporate donations influence how politicians communicate with voters during campaigns, even if they do not appear to affect their legislative priorities once elected.
\end{spacing}

\clearpage


%%---------------------------------------------------------------------------------
\bibliographystyle{plainnat}
\bibliography{biblio}

\clearpage


%%---------------------------------------------------------------------------------
\section{Figures}

\begin{figure}[htbp]
\centering
\includegraphics[width=0.85\textwidth]{Figure2a.pdf}
\caption{Candidate-level determinants of firm donations (reproduced from original paper, Figure~2a).}
\label{fig:figure2a}
\end{figure}

\begin{figure}[htbp]
\centering
\includegraphics[width=0.85\textwidth]{Figure2b.pdf}
\caption{Amount of firm donations (reproduced from original paper, Figure~2b).}
\label{fig:figure2b}
\end{figure}

\begin{figure}[htbp]
\centering
\includegraphics[width=0.85\textwidth]{Figure3.pdf}
\caption{Campaign revenue composition in 1993 and 1997 (reproduced from original paper, Figure~3).}
\label{fig:figure3}
\end{figure}

\begin{figure}[htbp]
\centering
\includegraphics[width=\textwidth]{Figure_robustness.pdf}
\caption{Robustness of Table~3 results across specifications. Each panel plots the coefficient on firm donations (loss) with 95\% confidence intervals for a different outcome variable. Specifications: (A)~original, (B)~winsorized at 1st/99th percentiles, (C)~no controls, (D)~no candidate controls, (E)~decomposed mandates, (F)~number of mandates, (G)~positive-donation candidates only.}
\label{fig:robustness}
\end{figure}

\clearpage


%%---------------------------------------------------------------------------------
\section{Tables}

%%%%%%%%%%%%%%%%%%%%%%%%%%%%%%%%%%%%%%%%%%%%%%%%%%%%%%%%%%%%%%%%%%%%
\begin{table}[!htbp]\centering
\caption{Impact of Firm Donations on Local versus National Campaigning: Winsorization Check}
\label{tab:don_local_nat_winsor}

%====================== Panel A ======================%
\bigskip
\textbf{Panel A. Original}\\[1mm]

{
\def\sym#1{\ifmmode^{#1}\else\(^{#1}\)\fi}
\begin{tabular}{lccc}
\hline \hline
                & \shortstack{Local\\index}
                & \shortstack{Local\\references}
                & \shortstack{National\\references} \\
\hline
Firm donations (loss)
                &   -0.130          &   -0.219          &    0.098  \\
                &   (0.029)         &   (0.053)         &   (0.055)        \\
                &   [0.000]         &   [0.000]         &   [0.078]        \\
Observations    &     2602           &     2602           &     2602         \\
Mean outcome before ban
                &    -0.652          &     1.375          &     3.031        \\
R2-Within       &     0.037          &     0.027          &     0.012        \\
\hline \hline
\end{tabular}
}

%====================== Panel B ======================%
\bigskip
\textbf{Panel B. Winsorizing}\\[1mm]

{
\def\sym#1{\ifmmode^{#1}\else\(^{#1}\)\fi}
\begin{tabular}{lccc}
\hline \hline
                & \shortstack{Local\\index}
                & \shortstack{Local\\references}
                & \shortstack{National\\references} \\
\hline
Firm donations (loss)
                &   -0.172         &   -0.276         &    0.145         \\
                &   (0.038)        &   (0.068)        &   (0.069)        \\
                &   [0.000]        &   [0.000]        &   [0.037]        \\
\(N\)           &     2602         &     2602         &     2602         \\
Mean outcome before ban
                &    -0.652        &     1.375        &     3.031        \\
R2-Within       &    0.0384        &     0.0273       &     0.0124       \\
\hline \hline
\end{tabular}
}
\vspace{1mm}
\begin{flushleft}
\footnotesize
\justifying
\textit{Notes}: Standard errors in parentheses; \textit{p}-values in brackets.
Panel~A uses the original sample.
Panel~B reports robustness results using firm donations winsorized at the 1st and 99th percentiles.
\end{flushleft}
\end{table}

%%%%%%%%%%%%%%%%%%%%%%%%%%%%%%%%%%%%%%%%%%%%%%%%%%%%%%%%%%%%%%%%%%%%%%
\begin{table}[H]
%\begin{threeparttable}
\caption{Impact of Firm Donations on Partisan Leaning}
\label{tab:don_lr_orig_nat}

%====================== Panel A ======================%
\bigskip
\textbf{Panel A. Original}\\[1mm]

{
\def\sym#1{\ifmmode^{#1}\else\(^{#1}\)\fi}
\begin{tabular}{l*{4}{c}}
\hline\hline
          & \shortstack{Left-right\\score} 
          & \shortstack{Extremeness} 
          & \shortstack{Originality\\index}
          & \shortstack{National party\\references} \\
\hline
Firm donations (loss) 
          &   -0.006        &    0.008        &   -0.015        &    0.026        \\
          &   (0.005)       &   (0.005)       &   (0.014)       &   (0.029)       \\
          &   [0.274]       &   [0.062]       &   [0.270]       &   [0.380]       \\
\(N\)      &    2602         &    2602         &    2096         &    2096         \\
Mean before ban
          &   -0.0371       &    0.861        &   -1.840        &    0.911        \\
R2-Within &    0.00567      &    0.00718      &    0.00259      &    0.00469      \\
\hline\hline
\end{tabular}
}

%====================== Panel B ======================%
\bigskip
\textbf{Panel B. Winsorizing}\\[1mm]

{
\def\sym#1{\ifmmode^{#1}\else\(^{#1}\)\fi}
\begin{tabular}{l*{4}{c}}
\hline\hline
          & \shortstack{Left-right\\score} 
          & \shortstack{Extremeness} 
          & \shortstack{Originality\\index}
          & \shortstack{National party\\references} \\
\hline
Firm donations (loss) 
          &   -0.008        &    0.011        &   -0.011        &    0.039        \\
          &   (0.007)       &   (0.006)       &   (0.018)       &   (0.036)       \\
          &   [0.231]       &   [0.071]       &   [0.532]       &   [0.286]       \\
\(N\)      &    2602         &    2602         &    2096         &    2096         \\
Mean before ban
          &   -0.0371       &    0.861        &   -1.840        &    0.911        \\
R2-Within &    0.00581      &    0.00720      &    0.00222      &    0.00479      \\
\hline\hline
\end{tabular}
}

\vspace{1mm}
\begin{flushleft}
\footnotesize
\justifying
\textit{Notes}: Standard errors in parentheses; \textit{p}-values in brackets.  
Panel~A uses the original sample.  
Panel~B reports robustness results using firm donations winsorized at the 1st and 99th percentiles.
\end{flushleft}

%\end{threeparttable}
\end{table}

%%%%%%%%%%%%%%%%%%%%%%%%%%%%%%%%%%%%%%%%%%%%%%%%%%%%%%%%%%%%%%%%%%%%%%%%
\begin{landscape}
\begin{table}[!htbp]\centering
%\begin{threeparttable}
\caption{Heterogeneity by Party}
\label{tab:interaction_original}
\textbf{Panel A. Original}\\[1mm]
{
\def\sym#1{\ifmmode^{#1}\else\(^{#1}\)\fi}

\begin{tabular}{l*{7}{c}}
\hline\hline
                &\shortstack{Local\\index}
                &\shortstack{Local\\references}
                &\shortstack{National\\references}
                &\shortstack{Left-right\\score}
                &\shortstack{Extremeness}
                &\shortstack{Originality\\index}
                &\shortstack{National party\\references} \\
\hline

Communist $\times$
                & -0.107 & -0.220 & 0.093 & -0.005 & 0.020 & -0.037 & 0.044 \\
Firm donations
                & (0.076) & (0.176) & (0.090) & (0.015) & (0.015) & (0.049) & (0.037) \\
                & [0.159] & [0.212] & [0.303] & [0.752] & [0.182] & [0.451] & [0.231] \\

Green $\times$
                & -3.466 & -0.860 & 7.395 & -0.532 & 0.521 & -1.203 & 5.374 \\
Firm donations
                & (0.373) & (0.304) & (0.883) & (0.097) & (0.108) & (0.287) & (0.807) \\
                & [0.000] & [0.005] & [0.000] & [0.000] & [0.000] & [0.000] & [0.000] \\

Socialist $\times$
                & -0.142 & -0.198 & 0.157 & -0.012 & 0.008 & 0.031 & 0.087 \\
Firm donations
                & (0.049) & (0.091) & (0.081) & (0.009) & (0.008) & (0.025) & (0.039) \\
                & [0.004] & [0.031] & [0.054] & [0.187] & [0.305] & [0.205] & [0.025] \\

Right $\times$
                & -0.115 & -0.198 & 0.067 & -0.003 & 0.002 & -0.035 & -0.006 \\
Firm donations
                & (0.038) & (0.063) & (0.080) & (0.006) & (0.006) & (0.017) & (0.042) \\
                & [0.003] & [0.002] & [0.402] & [0.621] & [0.736] & [0.044] & [0.880] \\

Far-right $\times$
                & -0.475 & -1.304 & -0.115 & 0.449 & 0.489 & 0.285 & -1.174 \\
Firm donations
                & (0.607) & (0.310) & (1.474) & (0.539) & (0.538) & (0.992) & (0.768) \\
                & [0.434] & [0.000] & [0.938] & [0.405] & [0.364] & [0.774] & [0.127] \\

Other $\times$
                & -0.248 & -0.534 & 0.112 & -0.001 & 0.058 &   &   \\
Firm donations
                & (0.142) & (0.195) & (0.294) & (0.022) & (0.020) &   &   \\
                & [0.081] & [0.006] & [0.704] & [0.979] & [0.003] &   &   \\

\hline
Observations    & 2602 & 2602 & 2602 & 2602 & 2602 & 2096 & 2096 \\
Mean outcome    & -0.652 & 1.375 & 3.031 & -0.037 & 0.861 & -1.840 & 0.911 \\
R2-Within       & 0.040 & 0.029 & 0.014 & 0.008 & 0.013 & 0.006 & 0.008 \\
\hline\hline
\end{tabular}
}

%\end{threeparttable}
\end{table}
\end{landscape}

\begin{landscape}
\begin{table}[!htbp]\centering
%\begin{threeparttable}
%\caption{Interaction Effects of Firm Donations and Party Ideology: Winsorized Sample}
\label{tab:interaction_winsor}
\ContinuedFloat
\caption*{(continued)}
\textbf{Panel B. Winsorizing}\\[1mm]

{
\def\sym#1{\ifmmode^{#1}\else\(^{#1}\)\fi}
\begin{tabular}{l*{7}{c}}
\hline\hline
                &\shortstack{Local\\index}
                &\shortstack{Local\\references}
                &\shortstack{National\\references}
                &\shortstack{Left-right\\score}
                &\shortstack{Extremeness}
                &\shortstack{Originality\\index}
                &\shortstack{National party\\references} \\
\hline

Communist $\times$
                & -0.100 & -0.203 & 0.106 & -0.006 & 0.022 & -0.033 & 0.050 \\
Firm donations
                & (0.089) & (0.195) & (0.120) & (0.020) & (0.020) & (0.064) & (0.050) \\
                & [0.261] & [0.299] & [0.377] & [0.785] & [0.290] & [0.613] & [0.312] \\

Green $\times$
                & -3.466 & -0.861 & 7.394 & -0.532 & 0.521 & -1.203 & 5.374 \\
Firm donations
                & (0.373) & (0.304) & (0.881) & (0.097) & (0.108) & (0.287) & (0.807) \\
                & [0.000] & [0.005] & [0.000] & [0.000] & [0.000] & [0.000] & [0.000] \\

Socialist $\times$
                & -0.162 & -0.222 & 0.185 & -0.014 & 0.009 & 0.048 & 0.108 \\
Firm donations
                & (0.061) & (0.115) & (0.101) & (0.012) & (0.010) & (0.030) & (0.044) \\
                & [0.008] & [0.054] & [0.069] & [0.249] & [0.409] & [0.112] & [0.014] \\

Right $\times$
                & -0.168 & -0.270 & 0.121 & -0.007 & 0.002 & -0.042 & -0.005 \\
Firm donations
                & (0.053) & (0.087) & (0.109) & (0.009) & (0.008) & (0.021) & (0.054) \\
                & [0.002] & [0.002] & [0.267] & [0.434] & [0.812] & [0.045] & [0.932] \\

Far-right $\times$
                & -0.477 & -1.305 & -0.108 & 0.449 & 0.489 & 0.289 & -1.171 \\
Firm donations
                & (0.607) & (0.310) & (1.474) & (0.539) & (0.538) & (0.992) & (0.768) \\
                & [0.432] & [0.000] & [0.942] & [0.405] & [0.364] & [0.771] & [0.128] \\

Other $\times$
                & -0.281 & -0.571 & 0.161 & 0.001 & 0.063 &   &   \\
Firm donations
                & (0.152) & (0.224) & (0.307) & (0.024) & (0.021) &   &   \\
                & [0.064] & [0.011] & [0.599] & [0.983] & [0.003] &   &   \\
\hline
Observations    & 2602 & 2602 & 2602 & 2602 & 2602 & 2096 & 2096 \\
Mean outcome    & -0.652 & 1.375 & 3.031 & -0.037 & 0.861 & -1.840 & 0.911 \\
R2-Within       & 0.041 & 0.029 & 0.015 & 0.008 & 0.012 & 0.006 & 0.008 \\
\hline\hline
\end{tabular}
}

\vspace{1mm}
\begin{flushleft}
\footnotesize
\justifying
\textit{Notes}: Standard errors in parentheses; \textit{p}-values in brackets.  
Panel~A uses the original sample.  
Panel~B reports robustness results using firm donations winsorized at the 1st and 99th percentiles.
\end{flushleft}
%\end{threeparttable}
\end{table}
\end{landscape}

%%%%%%%%%%%%%%%%%%%%%%%%%%%%%%%%%%%%%%%%%%%%%%%%%%%%%%%%%%%%%%%%%%%%%%%%%%%%%%%%%%%%%%
\begin{table}[!htbp]\centering
\caption{Impact of Firm Donations on Broad Policy Topics}
\label{tab:policyareas}

%====================== Panel A ======================%
\bigskip
\textbf{Panel A. Original}\\[1mm]

{
\def\sym#1{\ifmmode^{#1}\else\(^{#1}\)\fi}
\begin{tabular}{lcccc}
\hline\hline
                & \shortstack{Economic\\policy}
                & \shortstack{Social\\policy}
                & \shortstack{Homeland and\\administration}
                & \shortstack{Foreign\\policy} \\
\hline
Firm donations (loss)
                &   -1.213        &    1.324        &   -0.982        &    0.274        \\
                &   (0.563)       &   (0.584)       &   (0.563)       &   (0.137)       \\
                &   [0.032]       &   [0.024]       &   [0.082]       &   [0.046]       \\
Observations    &    2602         &    2602         &    2602         &    2602         \\
Mean outcome before ban
                &   23.507        &   36.203        &   19.243        &    4.244        \\
R2-Within       &    0.013        &    0.010        &    0.006        &    0.009        \\
\hline\hline
\end{tabular}
}

%====================== Panel B ======================%
\bigskip
\textbf{Panel B. Winsorizing}\\[1mm]

{
\def\sym#1{\ifmmode^{#1}\else\(^{#1}\)\fi}
\begin{tabular}{lcccc}
\hline\hline
                & \shortstack{Economic\\policy}
                & \shortstack{Social\\policy}
                & \shortstack{Homeland and\\administration}
                & \shortstack{Foreign\\policy} \\
\hline
Firm donations (loss)
                &   -1.729        &    1.486        &   -0.810        &    0.307        \\
                &   (0.746)       &   (0.796)       &   (0.740)       &   (0.181)       \\
                &   [0.021]       &   [0.062]       &   [0.275]       &   [0.089]       \\
Observations    &    2602         &    2602         &    2602         &    2602         \\
Mean outcome before ban
                &   23.507        &   36.203        &   19.243        &    4.244        \\
R2-Within       &    0.014        &    0.009        &    0.004        &    0.008        \\
\hline\hline
\end{tabular}
}

\vspace{1mm}
\begin{flushleft}
\footnotesize
\justifying
\textit{Notes}: Standard errors in parentheses; \textit{p}-values in brackets.  
Panel~A uses the original sample.  
Panel~B reports robustness results using firm donations winsorized at the 1st and 99th percentiles.
\end{flushleft}

\end{table}

%%%%%%%%%%%%%%%%%%%%%%%%%%%%%%%%%%%%%%%%%%%%%%%%%%%%%%%%%%%%%%%%%%%%%%%%%%%%%
\begin{table}[H]
\caption{Impact of Firm Donations on Legislative Activity and Discourse: Written questions to the government
}
\label{tab:questions}

%====================== Panel A ======================%
\bigskip
\textbf{Panel A. Original}\\[1mm]

{
\def\sym#1{\ifmmode^{#1}\else\(^{#1}\)\fi}
\begin{tabular}{lcccc}
\hline\hline
                & \shortstack{Number\\of questions}
                & \shortstack{Local\\index}
                & \shortstack{Local\\references}
                & \shortstack{National\\references} \\
\hline
Firm donations (loss)
                &    4.593        &   -0.079        &    0.015        &    0.050        \\
                &    (6.569)      &   (0.047)       &   (0.010)       &   (0.024)       \\
                &    [0.485]      &   [0.093]       &   [0.123]       &   [0.039]       \\
Observations    &     416         &     416         &     416         &     416         \\
Mean outcome    &   113.731       &    -0.880       &    0.188        &    0.708        \\
R2-Within       &    0.028        &    0.056        &    0.067        &    0.047        \\
\hline\hline
\end{tabular}
}

%====================== Panel B ======================%
\bigskip
\textbf{Panel B. Winsorizing}\\[1mm]

{
\def\sym#1{\ifmmode^{#1}\else\(^{#1}\)\fi}
\begin{tabular}{lcccc}
\hline\hline
                & \shortstack{Number\\of questions}
                & \shortstack{Local\\index}
                & \shortstack{Local\\references}
                & \shortstack{National\\references} \\
\hline
Firm donations (loss)
                &    9.981        &   -0.108        &    0.021        &    0.069        \\
                &   (10.429)      &   (0.065)       &   (0.012)       &   (0.032)       \\
                &    [0.340]      &   [0.101]       &   [0.089]       &   [0.034]       \\
Observations    &     416         &     416         &     416         &     416         \\
Mean outcome    &   113.731       &    -0.880       &    0.188        &    0.708        \\
R2-Within       &    0.031        &    0.056        &    0.068        &    0.048        \\
\hline\hline
\end{tabular}
}

\vspace{1mm}
\begin{flushleft}
\footnotesize
\justifying
\textit{Notes}: Standard errors in parentheses; \textit{p}-values in brackets. 
Panel~A uses the original sample. 
Panel~B reports robustness results using firm donations winsorized at the 1st and 99th percentiles.
\end{flushleft}


\end{table}

%%%%%%%%%%%%%%%%%%%%%%%%%%%%%%%%%%%%%%%%%%%%%%%%%%%%%%%
\begin{table}[H]
\caption{Impact of Firm Donations on Legislative Activity and Discourse: Debate interventions}
\label{tab:interventions}

%====================== Panel A ======================%
\bigskip
\textbf{Panel A. Original}\\[1mm]

{
\def\sym#1{\ifmmode^{#1}\else\(^{#1}\)\fi}
\begin{tabular}{lcccc}
\hline\hline
                & \shortstack{Number\\of interventions}
                & \shortstack{Local\\index}
                & \shortstack{Local\\references}
                & \shortstack{National\\references} \\
\hline
Firm donations (loss)
                &   -2.510        &    0.056        &    0.034        &   -0.071        \\
                &   (3.450)       &   (0.049)       &   (0.027)       &   (0.098)       \\
                &   [0.468]       &   [0.254]       &   [0.197]       &   [0.469]       \\
Observations    &     356         &     354         &     354         &     354         \\
Mean outcome    &    27.674       &    -1.876       &     0.241       &     3.832       \\
R2-Within       &     0.049       &     0.021       &     0.011       &     0.017       \\
\hline\hline
\end{tabular}
}

%====================== Panel B ======================%
\bigskip
\textbf{Panel B. Winsorizing}\\[1mm]

{
\def\sym#1{\ifmmode^{#1}\else\(^{#1}\)\fi}
\begin{tabular}{lcccc}
\hline\hline
                & \shortstack{Number\\of interventions}
                & \shortstack{Local\\index}
                & \shortstack{Local\\references}
                & \shortstack{National\\references} \\
\hline
Firm donations (loss)
                &   -2.667        &    0.075        &    0.053        &   -0.098        \\
                &   (5.430)       &   (0.069)       &   (0.052)       &   (0.131)       \\
                &   [0.624]       &   [0.276]       &   [0.308]       &   [0.455]       \\
Observations    &     356         &     354         &     354         &     354         \\
Mean outcome    &    27.674       &    -1.876       &     0.241       &     3.832       \\
R2-Within       &     0.048       &     0.020       &     0.012       &     0.017       \\
\hline\hline
\end{tabular}
}

\vspace{1mm}
\begin{flushleft}
\footnotesize
\justifying
\textit{Notes}: Standard errors in parentheses; \textit{p}-values in brackets.  
Panel~A uses the original sample.  
Panel~B reports robustness results using firm donations winsorized at the 1st and 99th percentiles.
\end{flushleft}

\end{table}

%%%%%%%%%%%%%%%%%%%%%%%%%%%%%%%%%%%%%%%%%%%%%%%%%%%%%%%

\begin{table}[H]\centering
\footnotesize 
\setlength{\tabcolsep}{8pt} 
\renewcommand{\arraystretch}{0.95}
\caption{Impact of Firm Donations on Local versus National Campaigning}
\label{tab:don_local_nat}

%====================== Panel A ======================%
\bigskip
\textbf{Panel A. Original}\\[1mm]

{
\def\sym#1{\ifmmode^{#1}\else\(^{#1}\)\fi}
\begin{tabular}{l*{3}{c}}
\hline\hline
                &\multicolumn{1}{c}{\shortstack{Local\\index}}&\multicolumn{1}{c}{\shortstack{Local\\references}}&\multicolumn{1}{c}{\shortstack{National\\references}}\\\cmidrule(lr){2-2}\cmidrule(lr){3-3}\cmidrule(lr){4-4}
                &\multicolumn{1}{c}{(1)}         &\multicolumn{1}{c}{(2)}         &\multicolumn{1}{c}{(3)}         \\
\hline
Firm donations (loss)&   -0.130\sym{***}&   -0.219\sym{***}&    0.098\sym{*}  \\
                &  (0.029)         &  (0.053)         &  (0.055)         \\
\hline
Observations    &     2602         &     2602         &     2602         \\
Mean outcome before ban&   -0.652         &    1.375         &    3.031         \\
R2-Within       &    0.037         &    0.027         &    0.012         \\
\hline\hline
\end{tabular}
}

%====================== Panel B ======================%
\bigskip
\textbf{Panel B. No controls}\\[1mm]

{
\def\sym#1{\ifmmode^{#1}\else\(^{#1}\)\fi}
\begin{tabular}{l*{3}{c}}
\hline\hline
                &\multicolumn{1}{c}{\shortstack{Local\\index}}&\multicolumn{1}{c}{\shortstack{Local\\references}}&\multicolumn{1}{c}{\shortstack{National\\references}}\\\cmidrule(lr){2-2}\cmidrule(lr){3-3}\cmidrule(lr){4-4}
                &\multicolumn{1}{c}{(1)}         &\multicolumn{1}{c}{(2)}         &\multicolumn{1}{c}{(3)}         \\
\hline
Firm donations (loss)&   -0.142\sym{***}&   -0.231\sym{***}&    0.116\sym{**} \\
                &  (0.027)         &  (0.045)         &  (0.052)         \\
\hline
Observations    &     2602         &     2602         &     2602         \\
Mean outcome before ban&   -0.652         &    1.375         &    3.031         \\
R2-Within       &    0.017         &    0.014         &    0.003         \\
\hline\hline
\end{tabular}
}

%====================== Panel C ======================%
\bigskip
\textbf{Panel C. No candidate controls}\\[1mm]

{
\def\sym#1{\ifmmode^{#1}\else\(^{#1}\)\fi}
\begin{tabular}{l*{3}{c}}
\hline\hline
                &\multicolumn{1}{c}{\shortstack{Local\\index}}&\multicolumn{1}{c}{\shortstack{Local\\references}}&\multicolumn{1}{c}{\shortstack{National\\references}}\\\cmidrule(lr){2-2}\cmidrule(lr){3-3}\cmidrule(lr){4-4}
                &\multicolumn{1}{c}{(1)}         &\multicolumn{1}{c}{(2)}         &\multicolumn{1}{c}{(3)}         \\
\hline
Firm donations (loss)&   -0.113\sym{***}&   -0.192\sym{***}&    0.082         \\
                &  (0.027)         &  (0.044)         &  (0.054)         \\
\hline
Observations    &     2602         &     2602         &     2602         \\
Mean outcome before ban&   -0.652         &    1.375         &    3.031         \\
R2-Within       &    0.025         &    0.018         &    0.006         \\
\hline\hline
\end{tabular}
}

%====================== Panel D ======================%
\bigskip
\textbf{Panel D. Breaking down ‘other mandate’: three distinct roles}\\[1mm]

{
\def\sym#1{\ifmmode^{#1}\else\(^{#1}\)\fi}
\begin{tabular}{l*{3}{c}}
\hline\hline
                &\multicolumn{1}{c}{\shortstack{Local\\index}}&\multicolumn{1}{c}{\shortstack{Local\\references}}&\multicolumn{1}{c}{\shortstack{National\\references}}\\\cmidrule(lr){2-2}\cmidrule(lr){3-3}\cmidrule(lr){4-4}
                &\multicolumn{1}{c}{(1)}         &\multicolumn{1}{c}{(2)}         &\multicolumn{1}{c}{(3)}         \\
\hline
Firm donations (loss)&   -0.131\sym{***}&   -0.220\sym{***}&    0.099\sym{*}  \\
                &  (0.029)         &  (0.053)         &  (0.055)         \\
\hline
Observations    &     2602         &     2602         &     2602         \\
Mean outcome before ban&   -0.652         &    1.375         &    3.031         \\
R2-Within       &    0.038         &    0.027         &    0.012         \\
\hline\hline
\end{tabular}
}

%====================== Panel E ======================%
\bigskip
\textbf{Panel E. Number of mandates instead of individual mandates}\\[1mm]

{
\def\sym#1{\ifmmode^{#1}\else\(^{#1}\)\fi}
\begin{tabular}{l*{3}{c}}
\hline\hline
                &\multicolumn{1}{c}{\shortstack{Local\\index}}&\multicolumn{1}{c}{\shortstack{Local\\references}}&\multicolumn{1}{c}{\shortstack{National\\references}}\\\cmidrule(lr){2-2}\cmidrule(lr){3-3}\cmidrule(lr){4-4}
                &\multicolumn{1}{c}{(1)}         &\multicolumn{1}{c}{(2)}         &\multicolumn{1}{c}{(3)}         \\
\hline
Firm donations (loss)&   -0.123\sym{***}&   -0.205\sym{***}&    0.095\sym{*}  \\
                &  (0.028)         &  (0.048)         &  (0.054)         \\
\hline
Observations    &     2602         &     2602         &     2602         \\
Mean outcome before ban&   -0.652         &    1.375         &    3.031         \\
R2-Within       &    0.033         &    0.022         &    0.011         \\
\hline\hline
\end{tabular}
}


\vspace{1mm}
\begin{flushleft}
\footnotesize
\justifying
\textit{Notes}: Standard errors are reported in parentheses; \textit{p}-values in brackets.  
Panel~A uses the original sample.  
Panel~B reports results without any controls.  
Panel~C omits candidate-specific controls.  
Panel~D breaks down the ``Other mandate'' variable into the three distinct roles: Conseiller départemental, Sénateur, and Député européen.  
Panel~E uses the total number of mandates instead of the four individual mandate variables.
\end{flushleft}
\end{table}


%%%%%%%%%%%%%%%%%%%%%%%%%%%%%%%%%%%%%%%%%%%%%%%%%%%%%%%%%%%%%%%%%%%%%%%%%%%%%
\begin{table}[!htbp]\centering
\caption{Impact of Firm Donations on Local versus National Campaigning: Positive-Donation Candidates}
\label{tab:don_local_nat_positive}

%====================== Panel A ======================%
\bigskip
\textbf{Panel A. Original}\\[1mm]

{
\def\sym#1{\ifmmode^{#1}\else\(^{#1}\)\fi}
\begin{tabular}{lccc}
\hline \hline
                & \shortstack{Local\\index}
                & \shortstack{Local\\references}
                & \shortstack{National\\references} \\
\hline
Firm donations (loss)
                &   -0.130          &   -0.219          &    0.098  \\
                &   (0.029)         &   (0.053)         &   (0.055)        \\
                &   [0.000]         &   [0.000]         &   [0.078]        \\
Observations    &     2602           &     2602           &     2602         \\
Mean outcome before ban
                &    -0.652          &     1.375          &     3.031        \\
R2-Within       &     0.037          &     0.027          &     0.012        \\
\hline \hline
\end{tabular}
}

%====================== Panel B ======================%
\bigskip
\textbf{Panel B. Positive-donation candidates only}\\[1mm]

{
\def\sym#1{\ifmmode^{#1}\else\(^{#1}\)\fi}
\begin{tabular}{lccc}
\hline \hline
                & \shortstack{Local\\index}
                & \shortstack{Local\\references}
                & \shortstack{National\\references} \\
\hline
Firm donations (loss)
                &   -0.118\sym{***}  &   -0.230\sym{***}  &    0.049         \\
                &   (0.032)          &   (0.062)          &   (0.056)        \\
Observations    &     1246           &     1246           &     1246         \\
Mean outcome before ban
                &    -0.246          &     2.050          &     2.672        \\
R2-Within       &     0.052          &     0.034          &     0.022        \\
\hline \hline
\end{tabular}
}

\vspace{1mm}
\begin{flushleft}
\footnotesize
\justifying
\textit{Notes}: Standard errors in parentheses; \textit{p}-values in brackets. \sym{*} \(p<0.10\), \sym{**} \(p<0.05\), \sym{***} \(p<0.01\).
Panel~A uses the original sample.
Panel~B restricts the sample to candidates who received strictly positive firm donations in 1993 ($N=1{,}246$, compared to $2{,}602$ in the full sample).
\end{flushleft}
\end{table}

%%%%%%%%%%%%%%%%%%%%%%%%%%%%%%%%%%%%%%%%%%%%%%%%%%%%%%%%%%%%%%%%%%%%%%%%%%%%%
\begin{table}[!htbp]\centering
\caption{Impact of Firm Donations on Partisan Leaning: Positive-Donation Candidates}
\label{tab:don_lr_positive}

%====================== Panel A ======================%
\bigskip
\textbf{Panel A. Original}\\[1mm]

{
\def\sym#1{\ifmmode^{#1}\else\(^{#1}\)\fi}
\begin{tabular}{l*{4}{c}}
\hline\hline
          & \shortstack{Left-right\\score}
          & \shortstack{Extremeness}
          & \shortstack{Originality\\index}
          & \shortstack{National party\\references} \\
\hline
Firm donations (loss)
          &   -0.006        &    0.008        &   -0.015        &    0.026        \\
          &   (0.005)       &   (0.005)       &   (0.014)       &   (0.029)       \\
          &   [0.274]       &   [0.062]       &   [0.270]       &   [0.380]       \\
\(N\)      &    2602         &    2602         &    2096         &    2096         \\
Mean before ban
          &   -0.0371       &    0.861        &   -1.840        &    0.911        \\
R2-Within &    0.00567      &    0.00718      &    0.00259      &    0.00469      \\
\hline\hline
\end{tabular}
}

%====================== Panel B ======================%
\bigskip
\textbf{Panel B. Positive-donation candidates only}\\[1mm]

{
\def\sym#1{\ifmmode^{#1}\else\(^{#1}\)\fi}
\begin{tabular}{l*{4}{c}}
\hline\hline
          & \shortstack{Left-right\\score}
          & \shortstack{Extremeness}
          & \shortstack{Originality\\index}
          & \shortstack{National party\\references} \\
\hline
Firm donations (loss)
          &   -0.003        &    0.006        &   -0.024\sym{*}  &    0.009        \\
          &   (0.005)       &   (0.005)       &   (0.015)        &   (0.031)       \\
\(N\)      &    1246         &    1246         &    1154          &    1154         \\
Mean before ban
          &    0.063        &    0.337        &   -1.407         &    0.812        \\
R2-Within &    0.007        &    0.017        &    0.016         &    0.019        \\
\hline\hline
\end{tabular}
}

\vspace{1mm}
\begin{flushleft}
\footnotesize
\justifying
\textit{Notes}: Standard errors in parentheses; \textit{p}-values in brackets. \sym{*} \(p<0.10\), \sym{**} \(p<0.05\), \sym{***} \(p<0.01\).
Panel~A uses the original sample.
Panel~B restricts the sample to candidates who received strictly positive firm donations in 1993.
Columns (3)--(4) have fewer observations because the originality index and national party references are defined only for candidates from the five major parties.
\end{flushleft}
\end{table}




\end{document} 